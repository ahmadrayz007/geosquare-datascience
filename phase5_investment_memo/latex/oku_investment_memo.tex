\documentclass[12pt,a4paper]{article}
\usepackage[margin=0.8cm]{geometry}
\usepackage{graphicx}
\usepackage[most]{tcolorbox}
\usepackage{xcolor}
\usepackage{fontawesome5}
\usepackage{multicol}

\definecolor{primaryblue}{RGB}{0,102,204}
\definecolor{accentgreen}{RGB}{46,125,50}

\pagestyle{empty}
\setlength{\parindent}{0pt}
\setlength{\parskip}{2pt}
\setlength{\columnsep}{12pt}

\begin{document}

% ========== PAGE 1: REGIONAL COMPARISON ==========

% Header
\begin{tcolorbox}[colback=primaryblue!5, colframe=primaryblue, boxrule=1pt, arc=2mm]
\begin{center}
\textbf{OKU Investment Analysis: Hidden Market Opportunity} | Grid-Based Spatial Intelligence
\end{center}
\end{tcolorbox}

\vspace{-2pt}

\begin{multicols}{2}

% LEFT COLUMN: Land Use Map
\begin{center}
\includegraphics[width=\columnwidth]{../outputs/maps/grid_land_use.png}
\end{center}

\vspace{-4pt}

\noindent\textbf{Land Use Reality}

OKU shows 68\% agricultural land vs Tangsel's 8\%. 96.5\% undeveloped (1.5M grids available). Massive greenfield opportunity with 22.6x more land area.

\vspace{4pt}

% LEFT COLUMN: Population + Night Lights
\begin{center}
\includegraphics[width=\columnwidth]{../outputs/maps/grid_population_density.png}
\end{center}

\vspace{-4pt}

\begin{center}
\includegraphics[width=\columnwidth]{../outputs/maps/grid_nightlights.png}
\end{center}

\vspace{-4pt}

\noindent\textbf{Economic Activity}

Tangsel: Dense urban (9,070/km²) with high economic activity (15.3 night lights). OKU: Rural (103/km²) with early-stage growth (0.8 night lights, 63K grids showing positive trend). Development gap = first-mover window.

\columnbreak

% RIGHT COLUMN: Business Gap Analysis
\begin{center}
\includegraphics[width=\columnwidth]{../business_gap_analysis/outputs/business_gap_analysis.png}
\end{center}

\vspace{-4pt}

\noindent\textbf{Business Landscape}

15.3x fewer businesses per capita. Critical gaps:
\begin{itemize}
\setlength\itemsep{0pt}
\item \textbf{Clinics}: 48x gap (390K/clinic vs 32K)
\item \textbf{Banks}: 19x gap (65K/bank vs 27K)
\item \textbf{Cafes}: 106x gap (0 locations!)
\item \textbf{Restaurants}: 28x gap
\item \textbf{Stores}: 14.5x gap
\end{itemize}

Near-zero competition, high unmet demand.

\vspace{4pt}

\noindent\textbf{Key Metrics (OKU)}

\scriptsize
\begin{tabular}{ll}
Area & 3,823 km² (22.6x) \\
Population & 393K \\
POI & 117 (34x gap) \\
Hazard & 0.28 (20.6\% safer) \\
Built & 2\% (98\% available) \\
Agricultural & 68\% \\
\end{tabular}

\end{multicols}

\clearpage

% ========== PAGE 2: INVESTMENT DECISION ==========

\begin{multicols}{2}

% LEFT COLUMN: RTRW Compliance Map
\begin{center}
\includegraphics[width=\columnwidth]{../../output/maps/rtrw_compliance_oku_only.png}
\end{center}

\vspace{-4pt}

\noindent\textbf{RTRW Compliance Analysis}

\textbf{93.4\% Compliance Rate} vs Tangsel's 66.4\%

\textbf{Green zones (1.39M grids, 93.4\%):} Aligned with government plans. Predictable market, low regulatory risk.

\textbf{Blue zones (50.9K grids, 3.4\%):} Residential-zoned but undeveloped. Land banking opportunity at rural pricing. Pre-approved zones not yet built = acquire before market realizes zoning advantage.

\textbf{Other discrepancies (3.2\%):} Forest→Developed (1.9\%), Agriculture→Built (0.5\%), Protected→Built (0.4\%), Industrial→Undeveloped (0.3\%).

\columnbreak

% RIGHT COLUMN: Investment Thesis
\begin{tcolorbox}[colback=accentgreen!5, colframe=accentgreen, boxrule=1pt]
\small
\textbf{Investment Thesis}

\textbf{OKU = First-Mover Advantage in Severely Underserved Market}
\end{tcolorbox}

\vspace{2pt}

\noindent\textbf{Strategy}

\textbf{1. SHORT-TERM (1-3 years)}\\
Essential services in Baturaja Timur
\begin{itemize}
\setlength\itemsep{0pt}
\item Clinics (48x gap = critical need)
\item Minimarkets (14.5x gap)
\item Banking/ATM (19x gap)
\end{itemize}

\textbf{2. MID-TERM (3-5 years)}\\
Agribusiness expansion
\begin{itemize}
\setlength\itemsep{0pt}
\item Oil palm processing (Kedaton P.R.)
\item Cold storage network
\item Farm-to-market logistics
\end{itemize}

\textbf{3. LONG-TERM (5-10 years)}\\
Land banking for development
\begin{itemize}
\setlength\itemsep{0pt}
\item Residential zones (50.9K grids)
\item 20.6\% lower disaster risk
\item Rural pricing before urbanization
\end{itemize}

\vspace{2pt}

\noindent\textbf{Top Kecamatan}

\scriptsize
\textbf{Agribusiness:} Kedaton Peninjauan Raya (100), Peninjauan (92)

\textbf{Retail/Healthcare:} Baturaja Timur (100), Baturaja Barat (52)

\end{multicols}

\vspace{2pt}

% Action Items (full width)
\noindent\textbf{Action Items}

\begin{multicols}{2}
\small
\begin{itemize}
\setlength\itemsep{0pt}
\item \faMapMarker\ Site visits: Baturaja Timur, Kedaton P.R., Sinar Peninjauan
\item \faFile\ Legal DD on RTRW settlement zones
\item \faUsers\ Partnerships: Agri cooperatives, Bupati office
\item \faChartLine\ Consumer surveys: Validate 48x clinic gap
\item \faDatabase\ Ground-truth POI data, financial modeling
\end{itemize}
\end{multicols}

\vfill

\begin{center}
\scriptsize
Grid analysis | 50m resolution | 1.5M cells | BNPB hazards, census, OSM POI, ESA WorldCover, VIIRS, RTRW | Dasymetric mapping (Stevens et al. 2015)
\end{center}

\end{document}
